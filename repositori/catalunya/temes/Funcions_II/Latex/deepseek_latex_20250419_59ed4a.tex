\section*{Problem 18}
\textbf{Enunciat}

2. S'han trobat unes pintures rupestres en una cova situada en una zona molt pedregosa. Hi ha un camí que voreja parcialment la cova format per l'arc de corba \( y = 4 - x^2 \) d'extrems (0, 4) i (2, 0). La cova està situada en el punt de coordenades (0, 2), tal com es mostra en la figura, i es vol habilitar un accés rectilini \( d \) des del camí a la cova que sigui el més curt possible.

a) Identifiqueu a la gràfica de la figura les coordenades de la cova i del punt del camí des d'on es vol habilitar l'accés. Comproveu que la funció \( f(x) = \sqrt{x^4 - 3x^2 + 4} \) calcula la distància des de cada punt del camí a la cova.

[1,25 punts]

b) Calculeu les coordenades del punt del camí que queda més a prop de la cova i digueu quina serà la longitud de l'accés \( d \).

[1,25 punts]

\section*{Problem 19}
\textbf{Enunciat}

a) Calculeu les coordenades del punt de la corba \( y = f(x) \) en què la recta tangent a la corba en aquest punt és horitzontal. Estudieu si aquest punt és un extrem relatiu i classifiqueu-lo.

[1,25 punts]

b) Calculeu l'àrea del recinte delimitat per la corba \( y = f(x) \), les rectes verticals \( x = 1 \) i \( x = e \) i l'eix de les abscisses.

[1,25 punts]

\section*{Problem 20}
\textbf{Enunciat}

6. Una empresa de ceràmica vol posar a la venda una rajola quadrada de 20 cm de costat pintada a dos colors, de manera que la superfície de cada color sigui la mateixa i que si es posen les rajoles l'una al costat de l'altra es vegi un dibuix continu (figura 1).

Per a fer-ho, l'empresa utilitza en cada rajola la funció \( f(x) = x^3 - 3x^2 + 2x + 1 \) enquadrada entre els punts de coordenades (0, 0), (0, 2), (2, 0) i (2, 2), tal com es mostra en la figura 2, i fa servir com a unitat de mesura el decímetre.

a) Justifiqueu que, efectivament, aquesta funció permet ajuntar les rajoles de manera contínua i derivable.

[1,25 punts]

b) Justifiqueu que aquesta funció divideix el quadrat esmentat en dues parts que tenen la mateixa superfície.

[1,25 punts]

\section*{Problem 21}
\textbf{Enunciat}

3. Sigui \( f(x) \) una funció derivable la gràfica de la qual passa pel punt (0, 1). La gràfica de la seva derivada, \( f'(x) \), és la que es mostra en la figura.

a) Calculeu l'equació de la recta tangent a la gràfica de la funció \( f(x) \) en el punt de la gràfica d'abscissa \( x = 0 \).

[1,25 punts]

b) Trobeu les abscisses dels punts singulars de la funció \( f(x) \) i classifiqueu-los.

[1,25 punts]

\section*{Problem 22}
\textbf{Enunciat}

5. Una empresa està treballant en el disseny d'unes càpsules de cafè. L'empresa ha construït la secció transversal de les càpsules inscrivint-la en una semicircumferència de radi 1, traçant a continuació una corda \( CD \) paral·lela al diàmetre \( AB \) i incorporant el punt \( E \) en el punt mitjà de l'arc \( CD \). D'aquesta manera queda traçat el pentàgon \( ACEDB \), tal com es mostra en la figura.

a) Expresseu en funció de \( x \) i \( h \) l'àrea del pentàgon \( ACEDB \).

[1,25 punts]

b) Quina ha de ser la distància (indicada en la figura per \( h \)) a què s'ha de situar la corda \( CD \) de \( AB \) per tal que l'àrea del pentàgon \( ACEDB \) sigui màxima?

[1,25 punts]