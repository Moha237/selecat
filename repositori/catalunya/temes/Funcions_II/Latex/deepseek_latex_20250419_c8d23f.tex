\section*{Problem 1}
\textbf{Enunciat}

1. Considereu la funció \( f(x) = 2 \frac{\ln x}{x} \), definida per a \( x > 0 \).

a) Estudieu-ne els màxims i els mínims, i les zones de creixement i de decreixement.

[ ] punt

b) Aquesta funció té asímptotes? Feu un esbós de la seva gràfica.

[ ] punt

c) Calculeu l'equació de la recta tangent a la gràfica de \( y = f(x) \) en el punt d'abscissa \( x = 1 \).

[0,5 punts]

\section*{Problem 2}
\textbf{Enunciat}

2. Sigui la funció \( f(x) = \frac{1}{x} \).

a) Calculeu l'equació de la recta tangent a la gràfica de la funció \( f \) en el punt d'abscissa  
   \( x = 2 \).  
   [0,75 punts]

b) Calculeu l'equació de la recta tangent a la gràfica de la funció \( f \) en el punt d'abscissa  
   \( x = k \), en què \( k \) és un nombre real positiu.  
   [0,75 punts]

c) Comproveu que, tal com es pot veure en la figura de sota, la recta de l'apartat \( b \) determina un triangle d'àrea constant amb els semieixos positius de coordenades. Calculeu aquesta àrea.  
   [1 punt]

\[f(x) = \frac{1}{x}\]

\section*{Problem 3}
\textbf{Enunciat}

4. En una carretera principal hi trobem el poble A. A 12 km del poble A, hi ha un encreuament O amb una carretera secundària que talla perpendicularment la carretera principal. A 9 km de l'encreuament, a la carretera secundària, hi trobem el poble B. Es vol construir una torre de comunicacions T en un punt de la carretera principal situat entre el poble A i l'encreuament O. Aquesta torre ha d'estar connectada amb cadascun dels dos pobles en línia recta per cable. Sabem que installar el cable entre la torre T i el poble B té un preu de 250 €/km i, en canvi, installar el cable entre la torre T i el poble A té un preu de 125 €/km. Determineu a quina distància de l'encreuament O a la carretera principal cal situar la torre T perquè el preu del cablejat sigui mínim i quin serà el valor d'aquest preu mínim.

[2,5 punts]

\section*{Problem 4}
\textbf{Enunciat}

5. Volem construir un petit cobert de fusta de 6 m³ de volum, en forma de prisma rectangular, adossat a la paret lateral d'una casa, per a guardar-hi llenya. Només cal construir, per tant, el sostre i tres parets (la paret del fons del cobert és la de la casa a la qual està adossat). A més, volem que el cobert mesuri el triple d'amplària que de fondària. Cada metre quadrat de paret té un cost de construcció de 30 € i el sostre costa 50 € per metre quadrat. Un cop construït el cobert, afegir-hi una porta té un cost fix de 35 €.

a) Comproveu que el cost de construcció del cobert ve donat per la funció

\[ C(x) = \frac{150}{x} + 150x^2 + 35 \], on \( x \) és la fondària del cobert en metres.

[1,25 punts]

b) Calculeu quines han de ser les dimensions del cobert per tal que el cost de construcció sigui mínim i justifiqueu la resposta. Quin és aquest cost?

[1,25 punts]

\section*{Problem 5}
\textbf{Enunciat}

6. Volem construir una peça metàl·lica que tingui per secció un trapezi isòsceles amb la base superior tres vegades més llarga que la base inferior. Els altres costats del trapezi fan 10 mm, tal com podeu observar en la figura següent:

a) Expresseu l'altura del trapezi en funció de la longitud \( x \) de la base inferior.

[0,5 punts]

b) Calculeu la longitud de la base inferior del trapezi de manera que l'àrea de la peça sigui màxima i trobeu el valor d'aquesta àrea màxima.

[2 punts]

\section*{Problem 6}
\textbf{Enunciat}

5. La Núria té un jardí rectangular i vol fer-hi un tancat (rectangular o quadrat) de 8 m² per al seu gos. Ha pensat de posar el tancat tocant al mur del jardí, tal com es mostra a la figura de la dreta, per estalviar-se així un dels quatre costats.

El preu de la tanca que vol fer servir és de 2,5 €/m.

a) Quines dimensions ha de tenir el tancat perquè el cost sigui mínim? Quin és aquest cost mínim?

[1,75 punts]

b) Si manteniu la forma rectangular o quadrada del tancat i feu que un dels vèrtexs del jardí coincideixi amb un vèrtex del tancat, quants euros us podeu estalviar? Raoneu com posaríeu el tancat i justifiqueu amb càlculs matemàtics les dimensions de la vostra proposta.

[0,75 punts]

\section*{Problem 7}
\textbf{Enunciat}

5. Per a cada punt \((x, y)\) de la corba \( y = e^{-2x} \), amb \( x > 0 \) i \( y > 0 \), considereu el rectangle amb vèrtexs als punts \((0, 0)\), \((x, 0)\), \((0, y)\) i \((x, y)\).

a) Comproveu que, d'entre tots aquests rectangles, el que té \( x = \frac{1}{2} \) és el d'àrea màxima. Quin és el valor d'aquesta àrea?

[1,5 punts]

b) Calculeu l'equació de la recta tangent a la funció \( y = e^{-2x} \) en el punt d'abscissa \( x = 0 \), i el seu punt de tall amb l'eix de les abscisses.

[1 punt]