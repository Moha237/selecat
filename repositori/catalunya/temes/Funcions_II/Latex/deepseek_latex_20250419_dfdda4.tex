\section*{Problem 8}
\textbf{Enunciat}

6. Considereu la funció \( f(x) = x^3 \).

a) Calculeu en quin punt del tercer quadrant la recta tangent a \( y = f(x) \) és paral·lela a la recta \( 3x - y = 4 \). Calculeu l'equació de la recta tangent a la gràfica en aquest punt i feu un dibuix aproximat de la gràfica de la funció i les dues rectes.

[1,25 punts]

b) Calculeu l'àrea de la regió delimitada per \( y = f(x) \) i la recta \( y = 3x + 2 \).

[1,25 punts]

\section*{Problem 9}
\textbf{Enunciat}

6. Al pati d'una escola es vol crear una àrea de joc de 30 m² per als més petits en forma de trapezi rectangular, de manera que la base més gran mesuri el doble que la base més petita, tal com mostra la figura, i que el costat oblic respecte a les bases (D) sigui tan curt com sigui possible.

a) Justifiqueu que se satisfan les relacions següents:  
\[ h = \frac{20}{x} \]  
i  
\[ D(x) = \sqrt{\frac{400}{x^2} + x^2} \]

b) Trobeu les dimensions del trapezi per a les quals la longitud del costat \( D \) és mínima.  
[1,5 punts]

\section*{Problem 10}
\textbf{Enunciat}

2. Considereu la funció \( f(x) = \frac{9}{x^2 + x - 2} \).

a) Determineu el domini, les possibles asímptotes, els extrems relatius i els intervals de creixement i decreixement de la funció.

[1,25 punts]

b) Calculeu l'equació general de la recta tangent a la funció \( f(x) \) en el punt d'abscissa \( x = 4 \). Representeu en un mateix gràfic la funció \( f(x) \) i la recta tangent.

[1,25 punts]

\section*{Problem 11}
\textbf{Enunciat}

4. A \( \mathbb{R}^2 \), considereu els triangles rectangles que tenen els vèrtexs en els punts \( O = (0, 0) \),  
\( A = (x, 0) \) i \( B = (0, y) \), amb \( x > 0 \) i \( y > 0 \), i en què la suma dels catets és 10.  

a) Expresseu l'àrea del triangle \( AOB \) en funció de \( x \). Per a quin valor de \( x \) l'àrea del triangle \( AOB \) és la més gran possible? Quin valor té aquesta àrea màxima?  
[1,25 punts]  

b) Expresseu la hipotenusa del triangle \( AOB \) en funció de \( x \). Per a quin valor de \( x \) la hipotenusa del triangle \( AOB \) és la més petita possible? Quin és aquest valor mínim?  
[1,25 punts]

\section*{Problem 12}
\textbf{Enunciat}

4. a) En la figura es mostra la gràfica de la funció \( f(x) \). Representeu de manera esquemàtica la gràfica de la funció derivada de \( f(x) \). Expliqueu el raonament que heu seguit.

[1,25 punts]

b) Calculeu els valors de \( a \) i \( b \) perquè la funció \( g(x) = ax^3 + bx^2 + 1 \) tingui un punt d'inflexió en \( x = \frac{1}{2} \) i la seva derivada en aquest punt sigui \( -\frac{3}{2} \).

[1,25 punts]

\section*{Problem 13}
\textbf{Enunciat}

6. Considereu la funció \( f(x) = \frac{x^3}{x - 2} \).

a) Estudieu si té punts crítics i, en cas que en tingui, justifiqueu de quin tipus són. Determineu també quins són els intervals de creixement i decreixement de la funció.

[1,5 punts]

b) Comproveu que l'equació \( f(x) = 0 \) té una única solució en l'interval \((-2, 1)\).

[1 punt]

\section*{Problem 14}
\textbf{Enunciat}

1. Considereu la paràbola \( y = 4 - x^2 \) i un valor \( a > 0 \).

a) Comproveu que l'equació de la recta tangent a la gràfica de la paràbola en el punt d'abscissa \( x = a \) és \( y = -2ax + a^2 + 4 \) i calculeu els punts de tall d'aquesta recta tangent amb els eixos de coordenades.

[1,25 punts]

b) Calculeu el valor de \( a > 0 \) perquè l'àrea del triangle determinat per aquesta recta tangent i els eixos de coordenades sigui mínima.

[1,25 punts]

\section*{Problem 15}
\textbf{Enunciat}

6. Resoleu les dues qüestions següents:
a) Sigui \( f(x) = 2x^3 + mx^2 + nx + p \) una funció que té dos extrems relatius en \( x = -3 \) i en \( x = 1 \) i que passa pel punt (3, 4). Calculeu els valors de \( m \), \( n \) i \( p \).

[1,25 punts]

b) Calculeu l'equació de la recta tangent a la funció  
\[ f(x) = \frac{1-x}{1+x} \]  
en \( x = -3 \).

[1,25 punts]

\section*{Problem 16}
\textbf{Enunciat}

1. Tracem la recta tangent a la funció \( f(x) = \frac{1}{x^2} + 1 \) per un punt \( P = (a, f(a)) \) del primer quadrant. Aquesta recta juntament amb els eixos de coordenades formen un triangle.

a) Comproveu que l'àrea d'aquest triangle, en funció de \( a \), ve donada per la funció
\[ g(a) = \frac{(a^2 + 3)^2}{4a} \].

[1,25 punts]

b) En quin punt \( P \) l'àrea del triangle és mínima? Calculeu aquest valor mínim.

[1,25 punts]

\section*{Problem 17}
\textbf{Enunciat}

4. Considereu la funció \( f(x) = \frac{ax^2 + b}{x} \), en què \( a \) i \( b \) són dos paràmetres reals. Calculeu els valors de \( a \) i \( b \) de manera que la funció \( f(x) \) tingui una asímptota obliqua de pendent 1 i un mínim en el punt de la gràfica d'abscissa \( x = 2 \).

[2,5 punts]