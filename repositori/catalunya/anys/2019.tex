\documentclass{article}
\usepackage{amsmath}

\begin{document}


\subsection*{Year: 2019}

\subsubsection*{Sèrie 1}

\textbf{1. Problem 1}
Les pàgines d'un llibre han de tenir cada una \( 600 \, \text{cm}^{2} \) de superfície, amb uns marges al voltant del text de 2 cm a la part inferior, 3 cm a la part superior i 2 cm a cada costat. Calculeu les dimensions de la pàgina que permeten la superfície impresa més gran possible.
[2 punts]

\textbf{2. Problem 2}
Considereu el sistema d'equacions lineals següent, que depèn del paràmetre real \( k \):
\[
\left\{\begin{array}{c}
x + 3 y + 2 z = -1 \\
x + k^{2} y + 3 z = 2 k \\
3 x + 7 y + 7 z = k - 3
\end{array}\right.
\]
\begin{enumerate}
    \item Discutiu el sistema per als diferents valors del paràmetre \( k \).
    [1 punt]
    \item Resoleu el sistema per al cas \( k = -1 \).
    [1 punt]
\end{enumerate}

\textbf{3. Problem 3}
Un dron es troba en el punt \( P = (2, -3, 1) \) i volem dirigir-lo en línia recta fins al punt més proper del pla d'equació \( \pi: 3 x + 4 z + 15 = 0 \).
\begin{enumerate}
    \item Calculeu l'equació de la recta, en forma paramètrica, que ha de seguir el dron. Quina distància ha de recórrer fins a arribar al pla?
    [1 punt]
    \item Trobeu les coordenades del punt del pla on arribarà el dron.
    [1 punt]
\end{enumerate}

\textbf{4. Problem 4}
Considereu la funció \( f(x) = \frac{2 x^{3} - 5 x + 4}{1 - x} \).
\begin{enumerate}
    \item Calculeu-ne el domini i estudieu-ne la continuïtat. Té cap asímptota vertical?
    [1 punt]
    \item Observeu que \( f(-2) = -\frac{2}{3}, f(0) = 4 \) i \( f(2) = -10 \). Raoneu si, a partir d'aquesta informació, podem deduir que l'interval \( (-2, 0) \) conté un zero de la funció. Podem deduir-ho per a l'interval \( (0, 2) \)? Trobeu un interval determinat per dos enters consecutius que contingui, com a mínim, un zero d'aquesta funció.
    [1 punt]
\end{enumerate}

\textbf{5. Problem 5}
Sigui la matriu \( \boldsymbol{M} = \left(\begin{array}{ll}1 & a \\ a & 0\end{array}\right) \), en què \( a \) és un paràmetre real.
\begin{enumerate}
    \item Calculeu per a quins valors del paràmetre \( a \) se satisfà la igualtat \( \boldsymbol{M}^{2} - \boldsymbol{M} - 2 \boldsymbol{I} = \mathbf{0} \), en què \( \boldsymbol{I} \) és la matriu identitat i \( \mathbf{0} \) és la matriu nulla, totes dues d'ordre 2.
    [1 punt]
    \item Fent servir la igualtat de l'apartat anterior, trobeu una expressió general per a calcular la matriu inversa de la matriu \( \boldsymbol{M} \) i, a continuació, calculeu la inversa de \( \boldsymbol{M} \) per al cas \( a = \sqrt{2} \).
    [1 punt]
\end{enumerate}

\textbf{6. Problem 6}
Considereu les funcions \( f(x) = x^{2} \) i \( g(x) = \frac{1}{x} \), i la recta \( x = \mathrm{e} \).
\begin{enumerate}
    \item Feu un esbós de la regió delimitada per les seves gràfiques i l'eix de les abscisses. Calculeu les coordenades del punt de tall de \( y = f(x) \) amb \( y = g(x) \).
    [1 punt]
    \item Calculeu l'àrea de la regió descrita en l'apartat anterior.
    [1 punt]
\end{enumerate}

\subsubsection*{Sèrie 4}

\textbf{1. Problem 1}
Volem construir un marc rectangular de fusta que delimiti una àrea de \( 2 \, \text{m}^{2} \). Sabem que el preu de la fusta és de \( 7,5 \, € / \text{m} \) per als costats horitzontals i de \( 12,5 \, € / \text{m} \) per als costats verticals. Determineu les dimensions que ha de tenir el rectangle perquè el cost total del marc sigui el mínim possible. Quin és aquest cost mínim?
[2 punts]

\textbf{2. Problem 2}
Siguin la recta \( r: \left\{\begin{array}{l}x = 2 \\ y - z = 1\end{array}\right. \) i el pla \( \pi: x - z = 3 \).
\begin{enumerate}
    \item Calculeu l'equació paramètrica de la recta que és perpendicular al pla \( \pi \) i que el talla en el mateix punt en què el talla la recta \( r \).
    [1 punt]
    \item Trobeu els punts de \( r \) que estan a una distància de \( \sqrt{8} \) unitats del pla \( \pi \).
    [1 punt]
\end{enumerate}

\textbf{3. Problem 3}
Considereu el sistema d'equacions lineals següent, que depèn del paràmetre real \( a \):
\[
\left\{\begin{array}{c}
a x + 7 y + 5 z = 0 \\
x + a y + z = 3 \\
y + z = -2
\end{array}\right.
\]
\begin{enumerate}
    \item Discutiu el sistema per als diferents valors del paràmetre \( a \).
    [1 punt]
    \item Resoleu el sistema per al cas \( a = 2 \).
    [1 punt]
\end{enumerate}

\textbf{4. Problem 4}
Considereu la funció \( f(x) \), que depèn dels paràmetres reals \( n \) i \( m \) i és definida per
\[
f(x) = \begin{cases}
\mathrm{e}^{x} & \text{si } x \leq 0 \\
\frac{x^{2}}{4} + n & \text{si } 0 < x \leq 2 \\
\frac{3 x}{2} + m & \text{si } x > 2
\end{cases}
\]
\begin{enumerate}
    \item Calculeu els valors de \( n \) i \( m \) perquè la funció sigui contínua a tot el conjunt dels nombres reals.
    [1 punt]
    \item Per al cas \( n = -4 \) i \( m = -6 \), calculeu l'àrea de la regió limitada per la gràfica de \( f(x) \), l'eix de les abscisses i les rectes \( x = 0 \) i \( x = 4 \).
    [1 punt]
\end{enumerate}

\textbf{5. Problem 5}
Considereu els plans \( \pi_{1}: 2 x + a y + z = 5 \), \( \pi_{2}: x + a y + z = 1 \) i \( \pi_{3}: 2 x + (a + 1) y + (a + 1) z = 0 \), en què \( a \) és un paràmetre real.
\begin{enumerate}
    \item Estudieu per a quins valors del paràmetre \( a \) els tres plans es tallen en un punt.
    [1 punt]
    \item Comproveu que per al cas \( a = 1 \) la interpretació geomètrica del sistema format per les equacions dels tres plans és la que es mostra en la imatge següent:
    [1 punt]
\end{enumerate}

\textbf{6. Problem 6}
Sabem que una funció \( f(x) \) és contínua i derivable a tots els nombres reals, que té com a segona derivada \( f''(x) = 6 x \) i que la recta tangent en el punt d'abscissa \( x = 1 \) és horitzontal.
\begin{enumerate}
    \item Determineu l'abscissa dels punts d'inflexió de la funció \( f \) i els intervals de concavitat i convexitat. Justifiqueu que la funció \( f \) té un mínim relatiu en \( x = 1 \).
    [1 punt]
    \item Sabent, a més, que la recta tangent en el punt d'abscissa \( x = 1 \) és \( y = 5 \), calculeu l'expressió de la funció \( f \).
    [1 punt]
\end{enumerate}

\end{document}