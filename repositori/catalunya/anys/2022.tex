\documentclass{article}
\usepackage{amsmath}

\begin{document}

\subsection*{Year: 2022}

\subsubsection*{Sèrie 2}

\textbf{1. Problem 1}
Sigui \( f'(x) = 3 x^{2} - 12 x \) la derivada d'una funció \( f(x) \).
\begin{enumerate}
    \item Si sabem que \( f(x) \) talla l'eix de les abscisses en \( x = 1 \), calculeu l'expressió de la funció \( f(x) \).
    [0,75 punts]
    \item Calculeu l'abscissa del punt d'inflexió de \( f(x) \) i estudieu la concavitat de la funció.
    [0,75 punts]
    \item Sabem que l'àrea del recinte limitat per la corba \( y = f''(x) \), l'eix de les abscisses i les rectes \( x = 0 \) i \( x = a \), amb \( a > 2 \), és \( 15 \, \text{u}^{2} \). Calculeu el valor de \( a \).
    [1 punt]
\end{enumerate}

\textbf{2. Problem 2}
Considereu el sistema d'equacions lineals següent, que depèn del paràmetre real \( a \):
\[
\left\{\begin{array}{l}
a x + 2 y + 3 z = 2 \\
2 x + a y + z = a \\
x + y + 4 z = 1
\end{array}\right.
\]
\begin{enumerate}
    \item Discutiu el sistema per als diferents valors del paràmetre \( a \).
    [1,5 punts]
    \item Resoleu, si és possible, el sistema per al cas \( a = 2 \).
    [1 punt]
\end{enumerate}

\textbf{3. Problem 3}
Sigui la recta \( r \) definida per l'expressió següent:
\[
r: \left\{\begin{array}{l}
x = 2 + \lambda \\
y = -1 + 3 \lambda \\
z = 3 + \lambda
\end{array}\right.
\]
\begin{enumerate}
    \item Determineu la posició relativa de la recta \( r \) respecte al pla \( \pi: x - 2 y + 4 z - 4 = 0 \). Si és paral·lela, calculeu la distància de \( r \) a \( \pi \), i si és secant, calculeu el punt de tall.
    [1,25 punt]
    \item Calculeu l'equació de la recta \( s \) perpendicular al pla \( \pi \) i que talla la recta \( r \) en un punt \( P \), la primera coordenada del qual és 5 vegades més gran que la segona.
    [1,25 punts]
\end{enumerate}

\textbf{4. Problem 4}
\begin{enumerate}
    \item Trobeu una funció polinòmica \( y = g(x) \) de grau 3 tal que talli l'eix de les ordenades en el punt \( (0, 5) \), que la recta tangent a \( y = g(x) \) en el punt d'abscissa \( x = 1 \) sigui horitzontal i que \( g''(x) = 2 x + 1 \).
    [1 punt]
    \item Comproveu que la funció \( f(x) = -x^{3} + 6 x^{2} - 16 \) té una arrel a \( x = 2 \) i que és estrictament creixent a l'interval \( (0, 4) \). Utilitzeu aquesta informació per a calcular l'àrea determinada per la funció \( f(x) \), l'eix de les abscisses i les rectes \( x = 0 \) i \( x = 4 \).
    [1,5 punts]
\end{enumerate}

\textbf{5. Problem 5}
Sigui la matriu \( \boldsymbol{X} = \left(\begin{array}{lll}a & 1 & 0 \\ 0 & b & 1 \\ 0 & 0 & c\end{array}\right) \), que depèn dels paràmetres \( a, b \) i \( c \).
\begin{enumerate}
    \item Calculeu les matrius \( \boldsymbol{X} \) tals que \( \boldsymbol{X}^{2} = \left(\begin{array}{lll}1 & 0 & 1 \\ 0 & 1 & 0 \\ 0 & 0 & 1\end{array}\right) \).
    [1 punt]
    \item Determineu els valors de \( a, b \) i \( c \) perquè la matriu inversa de \( \boldsymbol{X} \) sigui \( \boldsymbol{X}^{-1} = \left[\begin{array}{ccc}\frac{1}{2} & -\frac{1}{2} & -\frac{1}{2} \\ 0 & 1 & 1 \\ 0 & 0 & -1\end{array}\right] \).
    [1 punt]
\end{enumerate}

\textbf{6. Problem 6}
Al pati d'una escola es vol crear una àrea de joc de \( 30 \, \text{m}^{2} \) per als més petits en forma de trapezi rectangular, de manera que la base més gran mesuri el doble que la base més petita, tal com mostra la figura, i que el costat oblic respecte a les bases \( (D) \) sigui tan curt com sigui possible.
\begin{enumerate}
    \item Justifiqueu que se satisfan les relacions següents: \( h = \frac{20}{x} \) i \( D(x) = \sqrt{\frac{400}{x^{2}} + x^{2}} \).
    [1 punt]
    \item Trobeu les dimensions del trapezi per a les quals la longitud del costat \( D \) és mínima.
    [1,5 punts]
\end{enumerate}

\subsubsection*{Sèrie 5}

\textbf{1. Problem 1}
Siguin les matrius \( \boldsymbol{C} = \left(\begin{array}{cc}-1 & 1 \\ -1 & 0\end{array}\right) \) i \( \boldsymbol{A} = \left(\begin{array}{cc}2 & 0 \\ 3 & -1\end{array}\right) \).
\begin{enumerate}
    \item Comproveu que \( \boldsymbol{C}^{3} = \boldsymbol{I}_{2} \), en què \( \boldsymbol{I}_{2} \) és la matriu identitat d'ordre 2, i deduïu que la matriu \( \boldsymbol{C} \) és invertible i que \( \boldsymbol{C}^{-1} = \boldsymbol{C}^{2} \). Calculeu \( \boldsymbol{C}^{2022} \).
    [1,5 punts]
    \item Resoleu l'equació matricial \( \boldsymbol{C} \cdot \boldsymbol{X} = \boldsymbol{A} - 2 \boldsymbol{I}_{2} \).
    [1 punt]
\end{enumerate}

\textbf{2. Problem 2}
Considereu la funció \( f(x) = x^{3} \) i sigui \( a \) un nombre real estrictament positiu.
\begin{enumerate}
    \item Calculeu l'equació de la recta \( t \) tangent a la gràfica de la funció \( f \) en el punt d'abscissa \( x = a \). Trobeu el punt de tall de la recta \( t \) amb l'eix de les abscisses (en funció de \( a \)).
    [1,25 punts]
    \item Feu un esbós de la gràfica de la funció \( f \) i la recta \( t \). Calculeu el valor de \( a \) perquè l'àrea en el primer quadrant limitada per la funció \( f \), la recta \( t \) i l'eix de les abscisses sigui \( 108 \, \text{u}^{2} \).
    [1,25 punts]
\end{enumerate}

\textbf{3. Problem 3}
Considereu el sistema d'equacions lineals
\[
\left.\begin{array}{c}
2 x - y + 3 z = 0 \\
m y + (3 - m) z = -6 \\
2 x - y + m z = 6
\end{array}\right\}
\]
en què \( m \) és un paràmetre real.
\begin{enumerate}
    \item Discutiu el sistema per als diferents valors del paràmetre \( m \).
    [1,25 punts]
    \item Resoleu el sistema, si és possible, quan \( m = 0 \) i quan \( m = 3 \). En cada cas, doneu la posició relativa dels tres plans a \( \mathbb{R}^{3} \).
    [1,25 punts]
\end{enumerate}

\textbf{4. Problem 4}
A \( \mathbb{R}^{2} \), considereu els triangles rectangles que tenen els vèrtexs en els punts \( O = (0, 0) \), \( A = (x, 0) \) i \( B = (0, y) \), amb \( x > 0 \) i \( y > 0 \), i en què la suma dels catets és 10.
\begin{enumerate}
    \item Expresseu l'àrea del triangle \( A O B \) en funció de \( x \). Per a quin valor de \( x \) l'àrea del triangle \( A O B \) és la més gran possible? Quin valor té aquesta àrea màxima?
    [1,25 punts]
    \item Expresseu la hipotenusa del triangle \( A O B \) en funció de \( x \). Per a quin valor de \( x \) la hipotenusa del triangle \( A O B \) és la més petita possible? Quin és aquest valor mínim?
    [1,25 punts]
\end{enumerate}

\textbf{5. Problem 5}
Siguin els punts \( A = (0, 0, 1) \), \( B = (1, 1, 1) \), \( C = (-1, -1, 1) \) i \( D = (1, 0, 1) \).
\begin{enumerate}
    \item Comproveu que tres d'aquests punts estan alineats. Determineu quins són els tres punts i calculeu l'equació contínua i l'equació paramètrica de la recta que defineixen.
    [1,25 punts]
    \item Calculeu l'equació general o cartesiana del pla que determinen els quatre punts.
    [1,25 punts]
\end{enumerate}

\textbf{6. Problem 6}
La columna de l'esquerra de la taula següent mostra l'esquema d'un programa informàtic que s'ha elaborat per a trobar solucions aproximades d'una equació \( f(x) = 0 \) en un interval \( (a, b) \), sabent que \( f(a) \cdot f(b) < 0 \). La columna de la dreta recull un exemple de funcionament del programa en què es pot veure com actuaria per trobar una solució de l'equació \( x + \ln (x) = 0 \) entre els valors \( a = 0,5 \) i \( b = 2 \).
\begin{enumerate}
    \item Volem aplicar aquest programa per a trobar les tres arrels de \( f(x) = x^{3} - 3 x^{2} + 1 \) amb valors de \( a \) i \( b \) diferents. Trobeu justificadament entre quins valors \( a \) i \( b \), per a cada arrel, hem d'aplicar el programa per a trobar aproximacions de cadascuna de les tres arrels de la funció.
    [1,25 punts]
\end{enumerate}

\end{document}