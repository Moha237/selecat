\documentclass{article}
\usepackage{amsmath}

\begin{document}

\subsection*{Year: 2018}

\subsubsection*{Sèrie 1}

\textbf{1. Problem 1}
Siguin les matrius \( \boldsymbol{M} = \left(\begin{array}{cc}1 & 2 \\ 0 & -1 \\ t & 2\end{array}\right) \) i \( \boldsymbol{N} = \left(\begin{array}{ccc}-1 & t & 2 \\ 1 & 0 & -1\end{array}\right) \).
\begin{enumerate}
    \item Calculeu \( \boldsymbol{M} \cdot \boldsymbol{N} \) i comproveu que la matriu resultant no és invertible.
    [1 punt]
    \item Trobeu els valors de \( t \) per als quals la matriu \( \boldsymbol{N} \cdot \boldsymbol{M} \) és invertible.
    [1 punt]
\end{enumerate}

\textbf{2. Problem 2}
Sigui \( r \) la recta que passa pels punts \( A = (0, 1, 1) \) i \( B = (1, 1, -1) \).
\begin{enumerate}
    \item Trobeu l'equació paramètrica de la recta \( r \).
    [1 punt]
    \item Calculeu tots els punts de la recta \( r \) que estan a la mateixa distància dels plans \( \pi_{1}: x + y = -2 \) i \( \pi_{2}: x - z = 1 \).
    [1 punt]
\end{enumerate}

\textbf{3. Problem 3}
Sigui la funció \( f(x) = x^{3} - x^{2} \).
\begin{enumerate}
    \item Trobeu l'equació de la recta tangent a la gràfica i que és paral·lela a la recta d'equació \( x + 3 y = 0 \).
    [1 punt]
    \item Calculeu, si n'hi ha, els punts de la gràfica en què la funció presenta un màxim o mínim relatiu o un punt d'inflexió.
    [1 punt]
\end{enumerate}

\textbf{4. Problem 4}
Considereu els punts \( P = (3, -2, 1) \), \( Q = (5, 0, 3) \), \( R = (1, 2, 3) \) i la recta \( r: \left\{\begin{array}{l}x + y + 1 = 0 \\ 2 y + 3 z - 5 = 0\end{array}\right. \).
\begin{enumerate}
    \item Determineu l'equació general (és a dir, la que té la forma \( A x + B y + C z = D \)) del pla que passa per \( P \) i \( Q \) i és paral·lel a la recta \( r \).
    [1 punt]
    \item Donats el pla \( x + 2 y + m \cdot z = 7 \) i el pla que passa per \( P, Q \) i \( R \), trobeu \( m \) perquè siguin paral·lels i no coincidents.
    [1 punt]
\end{enumerate}

\textbf{5. Problem 5}
Sigui la funció \( f(x) = \sqrt{x} + x - 2 \).
\begin{enumerate}
    \item Comproveu que la funció \( f(x) \) compleix l'enunciat del teorema de Bolzano a l'interval \( [0, 2] \) i que, per tant, l'equació \( f(x) = 0 \) té alguna solució a l'interval \( (0, 2) \). Comproveu que \( x = 1 \) és una solució de l'equació \( f(x) = 0 \) i raoneu, tenint en compte el signe de \( f'(x) \), que la solució és única.
    [1 punt]
    \item A partir del resultat final de l'apartat anterior, trobeu l'àrea limitada per la gràfica de la funció \( f(x) \), l'eix de les abscisses i les rectes \( x = 0 \) i \( x = 1 \).
    [1 punt]
\end{enumerate}

\textbf{6. Problem 6}
Uns estudiants de batxillerat han programat un full de càlcul com el de la figura següent que dona la solució d'un sistema d'equacions compatible determinat d'una manera automàtica:
\begin{enumerate}
    \item Escriviu el sistema i comproveu que els valors proposats com a solució són correctes.
    [1 punt]
    \item Quin valor s'hauria de posar en lloc del 2 que està emmarcat en la imatge, corresponent a la cella E8 ( \( a_{33} \) de la matriu de coeficients), perquè el sistema fos incompatible?
    [1 punt]
\end{enumerate}

\subsubsection*{Sèrie 5}

\textbf{1. Problem 1}
Considereu el sistema d'equacions lineals \( \left\{\begin{array}{l}6 x + 3 y + 2 z = 5 \\ 3 x + 4 y + 6 z = 3, \text{ per a } m \in \mathbb{R} . \\ x + 3 y + 2 z = m\end{array}\right. \).
\begin{enumerate}
    \item Expliqueu raonadament que per a qualsevol valor del paràmetre \( m \) el sistema té una única solució.
    [1 punt]
    \item Resoleu el sistema i trobeu l'expressió general del punt solució.
    [1 punt]
\end{enumerate}

\textbf{2. Problem 2}
Siguin el pla d'equació \( \pi: x + y - z = 0 \) i el punt \( P = (2, 3, 2) \).
\begin{enumerate}
    \item Calculeu el punt simètric del punt \( P \) respecte del pla \( \pi \).
    [1 punt]
    \item Calculeu l'equació cartesiana (és a dir, la que té la forma \( A x + B y + C z = D \)) dels dos plans paral·lels al \( \pi \) que estan a una distància \( \sqrt{3} \) del punt \( P \).
    [1 punt]
\end{enumerate}

\textbf{3. Problem 3}
Sigui la funció \( f(x) = a \cdot \mathrm{e}^{-x^{2} + b x} \), amb \( a \neq 0 \) i \( b \neq 0 \).
\begin{enumerate}
    \item Calculeu els valors de \( a \) i de \( b \) que fan que la funció tingui un extrem relatiu en el punt \( (1, \mathrm{e}) \).
    [1 punt]
    \item Per al cas \( a = 3 \) i \( b = 5 \), calculeu l'asímptota horitzontal de la funció \( f \) quan \( x \) tendeix a \( +\infty \).
    [1 punt]
\end{enumerate}

\textbf{4. Problem 4}
Sabem que una funció \( f(x) \) està definida per a tots els nombres reals i que és derivable dues vegades. Sabem també que té un punt d'inflexió en el punt d'abscissa \( x = 2 \), que l'equació de la recta tangent a la gràfica de la funció \( f(x) \) en aquest punt és \( y = -124 x + 249 \) i que \( f(-3) = -4 \).
\begin{enumerate}
    \item Calculeu \( f''(2), f'(2) \) i \( f(2) \).
    [1 punt]
    \item Calculeu \( \int_{-3}^{2} f'(x) \, \mathrm{d} x \).
    [1 punt]
\end{enumerate}

\textbf{5. Problem 5}
Siguin les rectes \( r_{1}: x - 1 = \frac{y - 2}{-1} = z - 5 \) i \( r_{2}: (x, y, z) = (2 - 3 \lambda, -1 + \lambda, 2) \).
\begin{enumerate}
    \item Trobeu l'equació cartesiana (és a dir, la que té la forma \( A x + B y + C z = D \)) del pla que conté la recta \( r_{1} \) i és paral·lel a la recta \( r_{2} \).
    [1 punt]
    \item Digueu quina condició s'ha de complir perquè existeixi un pla que contingui la recta \( r_{1} \) i sigui perpendicular a la recta \( r_{2} \). Amb les rectes \( r_{1} \) i \( r_{2} \) de l'enunciat, comproveu si existeix un pla que contingui la recta \( r_{1} \) i sigui perpendicular a la recta \( r_{2} \).
    [1 punt]
\end{enumerate}

\textbf{6. Problem 6}
Considereu la matriu \( \boldsymbol{A} = \left(\begin{array}{ccc}-1 & 0 & 1 \\ 0 & -1 & 0 \\ 1 & 1 & 1\end{array}\right) \).
\begin{enumerate}
    \item Si \( \boldsymbol{I} = \left(\begin{array}{ccc}1 & 0 & 0 \\ 0 & 1 & 0 \\ 0 & 0 & 1\end{array}\right) \) és la matriu identitat d'ordre 3, calculeu per a quins valors de \( k \) la matriu \( \boldsymbol{A} + k \boldsymbol{I} \) té inversa. Trobeu, si existeix, la matriu inversa de \( \boldsymbol{A} - 2 \boldsymbol{I} \).
    [1 punt]
    \item Calculeu la matriu \( \boldsymbol{X} \) que satisfà l'equació \( \boldsymbol{X} \cdot \boldsymbol{A} + \boldsymbol{A}^{\mathrm{T}} = 2 \cdot \boldsymbol{X} \), en què \( \boldsymbol{A}^{\mathrm{T}} \) és la matriu transposada de la matriu \( \boldsymbol{A} \).
    [1 punt]
\end{enumerate}

\end{document}