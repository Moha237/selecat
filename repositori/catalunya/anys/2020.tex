\documentclass{article}
\usepackage{amsmath}

\begin{document}

\subsection*{Year: 2020}

\subsubsection*{Sèrie 1}

\textbf{1. Problem 1}
Tracem la recta tangent a la funció \( f(x) = \frac{1}{x^{2}} + 1 \) per un punt \( P = (a, f(a)) \) del primer quadrant. Aquesta recta juntament amb els eixos de coordenades formen un triangle.
\begin{enumerate}
    \item Comproveu que l'àrea d'aquest triangle, en funció de \( a \), ve donada per la funció
    \[
    g(a) = \frac{\left(a^{2} + 3\right)^{2}}{4 a}
    \]
    [1,25 punts]
    \item En quin punt \( P \) l'àrea del triangle és mínima? Calculeu aquest valor mínim.
    [1,25 punts]
\end{enumerate}

\textbf{2. Problem 2}
Considereu el sistema d'equacions lineals següent, que depèn del paràmetre real \( k \):
\[
\left\{\begin{array}{l}
5 x + y + 4 z = 19 \\
k x + 2 y + 8 z = 28 \\
5 x + y - k z = 23 + k
\end{array}\right.
\]
\begin{enumerate}
    \item Discutiu el sistema per als diferents valors del paràmetre \( k \).
    [1,25 punts]
    \item Resoleu, si és possible, el sistema per al cas \( k = 0 \).
    [1,25 punts]
\end{enumerate}

\textbf{3. Problem 3}
\begin{enumerate}
    \item Calculeu l'equació general del pla \( \pi \) que passa pel punt \( (8, 8, 8) \) i té com a vectors directors \( \boldsymbol{u} = (1, 2, -3) \) i \( \boldsymbol{v} = (-1, 0, 3) \).
    [1,25 punt]
    \item Determineu el valor del paràmetre \( a \) perquè el punt \( (1, -5, a) \) pertanyi al pla \( \pi \) i calculeu l'equació paramètrica de la recta que passa per aquest punt i és perpendicular al pla \( \pi \).
    [1,25 punts]
\end{enumerate}

\textbf{4. Problem 4}
Considereu la funció \( f(x) = \frac{a x^{2} + b}{x} \), en què \( a \) i \( b \) són dos paràmetres reals. Calculeu els valors de \( a \) i \( b \) de manera que la funció \( f(x) \) tingui una asímptota obliqua de pendent 1 i un mínim en el punt de la gràfica d'abscissa \( x = 2 \).
[2,5 punts]

\textbf{5. Problem 5}
Sigui la matriu \( \boldsymbol{A} = \left(\begin{array}{cc}1 & 1 \\ -3 & -4\end{array}\right) \).
\begin{enumerate}
    \item Trobeu la matriu \( \boldsymbol{X} \) que satisfà l'equació \( \boldsymbol{A X} = \boldsymbol{I} - 3 \boldsymbol{X} \), en què \( \boldsymbol{I} \) és la matriu identitat d'ordre 2.
    [1,25 punts]
    \item Comproveu que la matriu \( \boldsymbol{X} \) és invertible i calculeu-ne la matriu inversa.
    [1,25 punts]
\end{enumerate}

\textbf{6. Problem 6}
Considereu la funció \( f(x) = x^{3} \).
\begin{enumerate}
    \item Calculeu en quin punt del tercer quadrant la recta tangent a \( y = f(x) \) és paral·lela a la recta \( 3 x - y = 4 \). Calculeu l'equació de la recta tangent a la gràfica en aquest punt i feu un dibuix aproximat de la gràfica de la funció i les dues rectes.
    [1,25 punts]
    \item Calculeu l'àrea de la regió delimitada per \( y = f(x) \) i la recta \( y = 3 x + 2 \).
    [1,25 punts]
\end{enumerate}

\subsubsection*{Sèrie 3}

\textbf{1. Problem 1}
Sigui \( \boldsymbol{A} = \left(\begin{array}{ccc}1 & 0 & 2 \\ 1 & -1 & 1 \\ 0 & a & 1\end{array}\right) \), en què \( a \) és un paràmetre real.
\begin{enumerate}
    \item Determineu el rang de la matriu \( \boldsymbol{A} \) en funció del paràmetre \( a \).
    [1,25 punts]
    \item Comproveu que \( \operatorname{det}\left(\boldsymbol{A}^{2} + \boldsymbol{A}\right) = 0 \).
    [1,25 punt]
\end{enumerate}

\textbf{2. Problem 2}
S'han trobat unes pintures rupestres en una cova situada en una zona molt pedregosa. Hi ha un camí que voreja parcialment la cova format per l'arc de corba \( y = 4 - x^{2} \) d'extrems \( (0, 4) \) i \( (2, 0) \). La cova està situada en el punt de coordenades \( (0, 2) \), tal com es mostra en la figura, i es vol habilitar un accés rectilini \( d \) des del camí a la cova que sigui el més curt possible.
\begin{enumerate}
    \item Identifiqueu a la gràfica de la figura les coordenades de la cova i del punt del camí des d'on es vol habilitar l'accés. Comproveu que la funció \( f(x) = \sqrt{x^{4} - 3 x^{2} + 4} \) calcula la distància des de cada punt del camí a la cova.
    [1,25 punts]
    \item Calculeu les coordenades del punt del camí que queda més a prop de la cova i digueu quina serà la longitud de l'accés \( d \).
    [1,25 punts]
\end{enumerate}

\textbf{3. Problem 3}
Considereu el sistema d'equacions lineals següent:
\[
\left\{\begin{array}{c}
a x + y = a \\
x + a y + z = 5 \\
x + 2 y + z = 5
\end{array}\right.
\]
\begin{enumerate}
    \item Discutiu el sistema per als diferents valors del paràmetre \( a \).
    [1,25 punts]
    \item Resoleu el sistema per al cas \( a = 2 \).
    [1,25 punts]
\end{enumerate}

\textbf{4. Problem 4}
Sigui la funció \( f(x) = \frac{1}{x} \cdot \ln (x) \), en què \( \ln \) indica el logaritme neperià, definida per a \( x > 0 \).
\begin{enumerate}
    \item Calculeu les coordenades del punt de la corba \( y = f(x) \) en què la recta tangent a la corba en aquest punt és horitzontal. Estudieu si aquest punt és un extrem relatiu i classifiqueu-lo.
    [1,25 punts]
    \item Calculeu l'àrea del recinte delimitat per la corba \( y = f(x) \), les rectes verticals \( x = 1 \) i \( x = \mathrm{e} \) i l'eix de les abscisses.
    [1,25 punts]
\end{enumerate}

\textbf{5. Problem 5}
Considereu la recta \( r \) d'equació \( \frac{x - 1}{2} = \frac{y - 3}{-2} = \frac{z}{1} \) i la recta \( s \) que passa pel punt \( P = (2, -5, 1) \) i que té per vector director \( (-1, 0, -1) \).
\begin{enumerate}
    \item Estudieu la posició relativa de les rectes \( r \) i \( s \).
    [1,25 punt]
    \item Calculeu l'equació general del pla que és paral·lel a la recta \( r \) i conté la recta \( s \).
    [1,25 punts]
\end{enumerate}

\textbf{6. Problem 6}
Una empresa de ceràmica vol posar a la venda una rajola quadrada de \( 20 \, \text{cm} \) de costat pintada a dos colors, de manera que la superfície de cada color sigui la mateixa i que si es posen les rajoles l'una al costat de l'altra es vegi un dibuix continu (figura 1).
\begin{enumerate}
    \item Justifiqueu que, efectivament, aquesta funció permet ajuntar les rajoles de manera contínua i derivable.
    [1,25 punts]
    \item Justifiqueu que aquesta funció divideix el quadrat esmentat en dues parts que tenen la mateixa superfície.
    [1,25 punts]
\end{enumerate}

\end{document}