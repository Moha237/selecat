\documentclass{article}
\usepackage{amsmath}

\begin{document}

\subsection*{Year: 2024}

\subsubsection*{Sèrie 1}

\textbf{1. Problem 1}
Considereu la funció \( f(x) = 2 \frac{\ln x}{x} \), definida per a \( x > 0 \).
\begin{enumerate}
    \item Estudieu-ne els màxims i els mínims, i les zones de creixement i de decreixement. [1 punt]
    \item Aquesta función té asímptotes? Feu un esbós de la seva gràfica. [1 punt]
    \item Calculeu l'equació de la recta tangent a la gràfica de \( y = f(x) \) en el punt d'abscissa \( x = 1 \). [0,5 punt]
\end{enumerate}

\textbf{2. Problem 2}
Considereu el sistema d'equacions según:
\[
\left.\begin{array}{r}
4 x + 2 y - z = 4 \\
x - y + k z = 3 \\
3 x + 3 y = 1
\end{array}\right\}
\]
\begin{enumerate}
    \item Discutiu el sistema per als diferents valors del paràmetre \( k \), i resoleu-lo per a \( k = 0 \). [1 punt]
    \item Resoleu el sistema per a \( k = -1 \). [0,75 punts]
    \item Per a \( k = -1 \), modifiqueu la tercera equació de manera que el sistema esdevingui incompatible. Justifiqueu la resposta. [0,75 punt]
\end{enumerate}

\textbf{3. Problem 3}
En Joan troba entre els papers del seu avi un esbós com el de la figura adjunta, on es descriu un terreny de regadiu que ha deixat en herència al seu pare.
\begin{enumerate}
    \item A partir de l'expressió de \( f(x) \), calculeu les coordenades dels punts \( P, Q \) i \( R \) indicats a la figura. Calculeu també l'equació de la recta \( PR \). [1,25 punts]
    \item Calculeu la superficie del terreny. [1,25 punts]
\end{enumerate}

\textbf{4. Problem 4}
L'Andreu posa les nou boles que es mostren a continuació dins d'una bossa.
\begin{enumerate}
    \item A continuació, treu de la bossa dues boles a l'atzar, una darrere l'altra i sense reemplaçament (és a dir, no retorna a la bossa la primera bola abans de treure la segona).
    \begin{itemize}
        \item Calculeu la probabilitat que la primera bola sigui una A o una E. [0,5 punts]
        \item Calculeu la probabilitat que les dues boles siguin diferents. [0,75 punts]
    \end{itemize}
    \item L'Andreu torna a posar totes les boles a la bossa i en treu cinc a l'atzar, una darrere l'altra, però ara amb reemplaçament (és a dir, ara sí que retorna a la bossa cada bola extrema abans d'agafar la según).
    \begin{itemize}
        \item Calculeu la probabilitat que no hagi tret cap A. [0,5 punts]
        \item Calculeu la probabilitat que hagi tret almenys dues A. [0,75 punts]
    \end{itemize}
\end{enumerate}

\textbf{5. Problem 5}
Volem construir un petit cobert de fusta de \( 6 \, \text{m}^3 \) de volum, en forma de prisma rectangular, adossat a la paret lateral d'una casa, per a guardar-hi llenya. Només cal construir, per tant, el sostre i tres parets (la paret del fons del cobert és la de la casa a la qual està adossat). A més, volem que el cobert mesuri el triple d'amplària que de fondària. Cada metre quadrat de paret té un cost de construcció de \( 30 \, € \) i el sostre costa \( 50 \, € \) per metre quadrat. Un cop construït el cobert, afegir-hi una porta té un cost fix de \( 35 \, € \).
\begin{enumerate}
    \item Comproveu que el cost de construcció del cobert ve donat per la funció \( C(x) = \frac{300}{x} + 150 x^2 + 35 \), on \( x \) és la fondària del cobert en metres. [1,25 punts]
    \item Calculeu quines han de ser les dimensions del cobert per tal que el cost de construcció sigui mínim i justifiqueu la resposta. Quin és aquest cost? [1,25 punts]
\end{enumerate}

\textbf{6. Problem 6}
Considereu els punts \( A = (1, 2, 3) \) i \( B = (-3, -2, 3) \).
\begin{enumerate}
    \item Calculeu l'equació del pla \( \pi \) que és perpendicular a la recta \( AB \) i que passa pel punt mitjà entre \( A \) i \( B \). Justifiqueu que aquest pla està format, precisament, pels punts \( P = (x, y, z) \) que estan a igual distància de \( A \) que de \( B \), és a dir, \( d(P, A) = d(P, B) \). [1 punt]
    \item Calculeu les distàncies de \( A \) i de \( B \) al pla \( \pi \) i comproveu que són iguals. És casualitat? Raoneu la resposta. [0,75 punts]
    \item Sigui \( C = (-7, 6, 3) \). El triangle \( ABC \) és isòsceles? Calculeu la seva àrea. [0,75 punts]
\end{enumerate}

\subsubsection*{Sèrie 5}

\textbf{1. Problem 1}
Considereu la funció \( f(x) = -2 + 10(x-1) \ln x \), definida per a \( x > 0 \).
\begin{enumerate}
    \item Comproveu que \( f(x) \) té una arrel a l'interval \([1, 1.5]\) i busqueu un interval d'una dècima de longitud que també contingui aquesta mateixa arrel. [0,75 punts]
    \item Sense calcular els punts crítics, justifiqueu que \( f(x) \) és decreixent a l'interval \((0, 1)\) i creixent a \((1, +\infty)\). Quins màxims i mínims té aquesta funció? [1 punt]
    \item Calculeu \(\lim_{x \rightarrow 0^{+}} f(x)\) i \(\lim_{x \rightarrow +\infty} f(x)\), i feu un esbós de la gràfica d'aquesta función. [0,75 punts]
\end{enumerate}

\textbf{2. Problem 2}
Considereu les matrius \(\boldsymbol{P} = \left(\begin{array}{ccc}2 & 1 & -1 \\ 2 & 1 & 0 \\ 1 & 0 & 1\end{array}\right)\), \(\boldsymbol{Q} = \left(\begin{array}{lll}2 & 2 & 2 \\ 2 & 2 & 2 \\ 2 & 2 & 2\end{array}\right)\) i \(\boldsymbol{R} = \left(\begin{array}{lll}1 & 2 & 3 \\ 1 & 1 & 1 \\ 3 & 2 & 1\end{array}\right)\).
\begin{enumerate}
    \item Decidiu si la matriu \(\boldsymbol{P}\) és invertible i, en cas de ser-ho, calculeu la seva inversa. Expliqueu detalladament el procediment seguit. [1,25 punts]
    \item Calculeu una matriu \(\boldsymbol{X}\) de 3 files i 3 columnes que compleixi \(\boldsymbol{P} \boldsymbol{X} + \boldsymbol{Q} = 2 \boldsymbol{R}\). [1,25 punts]
\end{enumerate}

\textbf{3. Problem 3}
Considereu les paràboles \( y = f_a(x) \), amb \( f_a(x) = a x^2 + 2 x + 5 - a \), on \( a \) és un paràmetre real.
\begin{enumerate}
    \item Determineu el valor del paràmetre \( a \) per al qual la recta tangent a \( y = f_a(x) \) en el punt d'abscissa \( x = 1 \) passa pel punt \((2, 13)\). [1 punt]
    \item Calculeu els punts de tall de les paràboles \( y = f_1(x) \) i \( y = f_3(x) \). [0,5 punts]
    \item Calculeu l'àrea de la región situada entre les dues paràboles \( y = f_1(x) \) i \( y = f_3(x) \). [1 punt]
\end{enumerate}

\textbf{4. Problem 4}
La Rut fa servir el mètode següent per a fer els problemes de matemàtiques: tira un dau equilibrat i, si el resultat és com a màxim 4, pensa i resol el problema ella mateixa; si el resultat és 5 o 6, busca la solució del problema per Internet i la copia. Quan és ella qui ha pensat la solució, la resposta és correcta en el \( 75\% \) dels casos; quan copia la solució d'Internet, la resposta és correcta només en el \( 40\% \) dels casos.
\begin{enumerate}
    \item Quina és la probabilitat que la solució d'un problema respost seguint aquest mètode sigui correcta? [0,75 punts]
    \item Quina és la probabilitat que un problema l'hagi resolt la Rut si sabem que la solució és correcta? [0,75 punts]
    \item Demà la Rut ha d'entregar 5 problemes de matemàtiques. Quina és la probabilitat que n'hi hagi almenys 4 de correctes? [1 punt]
\end{enumerate}

\textbf{5. Problem 5}
En Carles vol construir un decorat per a l'obra de teatre de final de curs en forma d'un rectangle i dos semicercles, tal com es mostra a la figura següent:
\begin{enumerate}
    \item Determineu el perímetre i l'àrea del decorat que s'ha de construir en funció de \( x \) i de \( y \). [1 punt]
    \item Per a revestir el perímetre del decorat, en Carles té material per a cobrir fins a 10 m. Si el vol gastar tot, quines seran les mides del decorat d'àrea màxima que podrà construir? Quin és el valor d'aquesta àrea? [1,5 punts]
\end{enumerate}

\textbf{6. Problem 6}
Considereu les rectes \( r: \frac{x-5}{4} = \frac{y-4}{3} = \frac{z-3}{-1} \) i \( s: \left\{\begin{array}{l}x = 4 + 2k \\ y = 3 + k \\ z = -1\end{array}\right. \).
\begin{enumerate}
    \item Quina és la seva posició relativa? Calculeu l'equació implícita d'un pla \( \pi \) que sigui paral·lel a les dues rectes i que passi per l'origen de coordenades. [1,25 punts]
    \item Calculeu l'equació de la recta \( t \) que talla les dues rectes \( r \) i \( s \) perpendicularment. [1,25 punts]
\end{enumerate}

\end{document}