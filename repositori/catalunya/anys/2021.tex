\documentclass{article}
\usepackage{amsmath}

\begin{document}

\subsection*{Year: 2021}

\subsubsection*{Sèrie 2}

\textbf{1. Problem 1}
Considereu la paràbola \( y = 4 - x^{2} \) i un valor \( a > 0 \).
\begin{enumerate}
    \item Comproveu que l'equació de la recta tangent a la gràfica de la paràbola en el punt d'abscissa \( x = a \) és \( y = -2 a x + a^{2} + 4 \) i calculeu els punts de tall d'aquesta recta tangent amb els eixos de coordenades.
    [1,25 punts]
    \item Calculeu el valor de \( a > 0 \) perquè l'àrea del triangle determinat per aquesta recta tangent i els eixos de coordenades sigui mínima.
    [1,25 punts]
\end{enumerate}

\textbf{2. Problem 2}
Considereu el sistema d'equacions lineals següent, que depèn del paràmetre real \( p \):
\[
\left\{\begin{array}{r}
p x + y + z = 2 \\
2 x + p y + p^{2} z = 1 \\
2 x + y + z = 2
\end{array}\right.
\]
\begin{enumerate}
    \item Discutiu el sistema per als diferents valors del paràmetre \( p \).
    [1,5 punts]
    \item Resoleu, si és possible, el sistema per al cas \( p = 2 \).
    [1 punt]
\end{enumerate}

\textbf{3. Problem 3}
Considereu el punt \( P = (-1, 3, 1) \), el pla \( \pi: x = y \) i la recta \( r: \frac{x - 1}{2} = \frac{y}{3} = z - 2 \).
\begin{enumerate}
    \item Trobeu les coordenades del punt \( P' \) simètric a \( P \) respecte al pla \( \pi \).
    [1,25 punts]
    \item De tots els plans que contenen la recta \( r \), trobeu l'equació cartesiana del que és perpendicular al pla \( \pi \).
    [1,25 punts]
\end{enumerate}

\textbf{4. Problem 4}
Sigui la funció \( f(x) = \frac{\ln (x)}{x} \) definida en el domini \( x > 0 \), en què \( \ln \) és el logaritme neperià.
\begin{enumerate}
    \item Trobeu les coordenades d'un punt de la corba \( y = f(x) \) en el qual la recta tangent a la corba sigui horitzontal i analitzeu si la funció té un extrem relatiu en aquest punt.
    [1 punt]
    \item Determineu si la funció \( f(x) \) té alguna asímptota horitzontal.
    [0,5 punts]
    \item Calculeu l'àrea de la regió delimitada per la corba \( y = f(x) \) i les rectes \( x = 1 \) i \( x = \mathrm{e} \). Feu un dibuix aproximat de la gràfica de la funció en el domini \( 0 < x < 5 \), en què quedi representada l'àrea que heu calculat.
    [1 punt]
\end{enumerate}

\textbf{5. Problem 5}
\begin{enumerate}
    \item Donada la matriu \( \boldsymbol{A} = \left(\begin{array}{lll}0 & 0 & 1 \\ 1 & 0 & 0 \\ 0 & 1 & 0\end{array}\right) \), resoleu l'equació matricial \( \boldsymbol{A}^{2} \boldsymbol{X} = \boldsymbol{A} - 3 \boldsymbol{I} \), en què \( \boldsymbol{I} \) és la matriu identitat.
    [1,25 punts]
    \item Una matriu quadrada \( \boldsymbol{M} \) satisfà que \( \boldsymbol{M}^{3} - 3 \boldsymbol{M}^{2} + 3 \boldsymbol{M} - \boldsymbol{I} = 0 \), en què \( \boldsymbol{I} \) és la matriu identitat. Justifiqueu que \( \boldsymbol{M} \) és invertible i expresseu la inversa de \( \boldsymbol{M} \) en funció de les matrius \( \boldsymbol{M} \) i \( \boldsymbol{I} \).
    [1,25 punts]
\end{enumerate}

\textbf{6. Problem 6}
Considereu la funció \( f(x) = \mathrm{e}^{x-1} - x - 1 \).
\begin{enumerate}
    \item Estudieu-ne la continuïtat, els extrems relatius i els intervals de creixement i decreixement.
    [1,25 punts]
    \item Demostreu que l'equació \( f(x) = 0 \) té exactament dues solucions entre \( x = -1 \) i \( x = 3 \).
    [1,25 punts]
\end{enumerate}

\subsubsection*{Sèrie 5}

\textbf{1. Problem 1}
Considereu les matrius \( \boldsymbol{A} = \left(\begin{array}{cc}2 & 1 \\ -3 & 4\end{array}\right) \), \( \boldsymbol{B} = \left(\begin{array}{ll}1 & 1 \\ 1 & 0\end{array}\right) \) i \( \boldsymbol{C} = \left(\begin{array}{ll}5 & 1 \\ 0 & 7\end{array}\right) \).
\begin{enumerate}
    \item Raoneu que la matriu \( \boldsymbol{B} \) és invertible i després calculeu \( \boldsymbol{B}^{-1} \).
    [1,25 punts]
    \item Calculeu la matriu \( \boldsymbol{X} \) que satisfà la igualtat \( \boldsymbol{A} + \boldsymbol{B} \cdot \boldsymbol{X} = \boldsymbol{C} \cdot \boldsymbol{A} \).
    [1,25 punts]
\end{enumerate}

\textbf{2. Problem 2}
Siguin les funcions \( f(x) = x^{3} - 9 x \) i \( g(x) = 7 x \).
\begin{enumerate}
    \item Estudieu els intervals de creixement i decreixement de \( f(x) \).
    [1,25 punts]
    \item Calculeu l'àrea de la regió semiplà \( x \geq 0 \) compresa entre les gràfiques de \( f(x) \) i \( g(x) \).
    [1,25 punts]
\end{enumerate}

\textbf{3. Problem 3}
Considereu els punts de l'espai tridimensional \( A = (1, a, 1) \), \( B = (a, 1, 2) \), \( C = (1, 1, 1) \) i \( D = (0, 0, 0) \), en què \( a \) és un paràmetre real.
\begin{enumerate}
    \item Determineu el valor del paràmetre \( a \) per al qual els punts són diferents i coplanaris (és a dir, que hi ha un pla que els conté).
    [1,25 punts]
    \item Per al valor \( a = 2 \), calculeu l'àrea del triangle de vèrtexs \( A, B \) i \( C \).
    [1,25 punts]
\end{enumerate}

\textbf{4. Problem 4}
La resistència al trencament \( R \) d'una biga de secció rectangular de base \( x \) i altura \( y \) és directament proporcional al producte \( x y^{2} \); per tant, \( R = k x y^{2} \), en què \( k \) és una constant positiva. Disposem d'un tronc de fusta en forma de cilindre de diàmetre \( d \) com el de la figura.
\begin{enumerate}
    \item Comproveu que la resistència \( R \) de la biga rectangular de base \( x \) que podem construir amb aquest tronc ve donada per l'expressió \( R = k x \left(d^{2} - x^{2}\right) \).
    [1,25 punts]
    \item Calculeu les dimensions de la biga rectangular de resistència màxima que podem construir a partir d'aquest tronc i calculeu aquesta resistència màxima.
    [1,25 punts]
\end{enumerate}

\textbf{5. Problem 5}
Considereu el sistema d'equacions lineals següent, que depèn del paràmetre real \( a \):
\[
\left\{\begin{array}{c}
x + 2 y + a z = 8 \\
2 x + y - a z = 1 \\
3 x - 3 a z = 1
\end{array}\right.
\]
\begin{enumerate}
    \item Comproveu que, per a qualsevol valor del paràmetre \( a \), el sistema d'equacions lineals no té solució.
    [1,25 punts]
    \item Interpreteu geomètricament el sistema d'equacions lineals. Feu un dibuix esquemàtic que representi la posició relativa dels tres plans.
    [1,25 punts]
\end{enumerate}

\textbf{6. Problem 6}
Resoleu les dues qüestions següents:
\begin{enumerate}
    \item Sigui \( f(x) = 2 x^{3} + m x^{2} + n x + p \) una funció que té dos extrems relatius en \( x = -3 \) i en \( x = 1 \) i que passa pel punt \( (3, 4) \). Calculeu els valors de \( m, n \) i \( p \).
    [1,25 punts]
    \item Calculeu l'equació de la recta tangent a la funció \( f(x) = \frac{1 - x}{1 + x} \) en \( x = -3 \).
    [1,25 punt]
\end{enumerate}

\end{document}