\documentclass{article}
\usepackage{amsmath}

\begin{document}

\subsection*{Year: 2016}

\subsubsection*{Sèrie 3}

\textbf{1. Problem 1}
Considereu el sistema d'equacions lineals següent:
$$
\left.\begin{array}{rl}
2 x + 4 y + 4 z & = 4 k - 7 \\
2 x - k y & = -1 \\
-2 x & = k + 1
\end{array}\right\}
$$
\begin{enumerate}
    \item Discutiu el sistema per als diferents valors del paràmetre real $ k $.
    [1 punt]
    \item Resoleu el sistema per al cas $ k = 0 $.
    [1 punt]
\end{enumerate}

\textbf{2. Problem 2}
A $ \mathbb{R}^{3} $, siguin la recta $ r $ que té per equació $ (x, y, z) = (1 + \lambda, \lambda, 1 - \lambda) $ i el pla $ \pi $ d'equació $ 2 x - y + z = -2 $.
\begin{enumerate}
    \item Determineu la posició relativa de la recta $ r $ i el pla $ \pi $.
    [1 punt]
    \item Calculeu la distància entre la recta $ r $ i el pla $ \pi $.
    [1 punt]
\end{enumerate}

\textbf{3. Problem 3}
Sigui la funció $ f(x) = x \mathrm{e}^{x-1} $.
\begin{enumerate}
    \item Calculeu l'equació de la recta tangent a la gràfica de la funció $ f $ en el punt d'abscissa $ x = 1 $.
    [1 punt]
    \item Determineu en quins intervals la funció $ f $ és creixent i en quins intervals és decreixent.
    [1 punt]
\end{enumerate}

\textbf{4. Problem 4}
\begin{enumerate}
    \item Calculeu totes les matrius de la forma $ \boldsymbol{A} = \left(\begin{array}{cc}1 & 0 \\ m & -2\end{array}\right) $ que satisfan la igualtat $ \boldsymbol{A}^{2} + \boldsymbol{A} = 2 \boldsymbol{I} $, en què $ \boldsymbol{I} $ és la matriu identitat, $ \boldsymbol{I} = \left(\begin{array}{ll}1 & 0 \\ 0 & 1\end{array}\right) $.
    [1 punt]
    \item Justifiqueu que si $ \boldsymbol{A} $ és una matriu quadrada que compleix la igualtat $ \boldsymbol{A}^{2} + \boldsymbol{A} = 2 \boldsymbol{I} $, aleshores $ \boldsymbol{A} $ és invertible, i calculeu l'expressió de $ \boldsymbol{A}^{-1} $ en funció de les matrius $ \boldsymbol{A} $ i $ \boldsymbol{I} $.
    [1 punt]
\end{enumerate}

\textbf{5. Problem 5}
Considereu el tetraedre que té per vèrtexs els punts $ A = (x, 0, 1) $, $ B = (0, x, 1) $, $ C = (3, 0, 0) $ i $ D = (0, x, 0) $, amb $ 0 < x < 3 $.
\begin{enumerate}
    \item Comproveu que el volum del tetraedre és donat per l'expressió $ V(x) = \frac{1}{6} \left( -x^{2} + 3 x \right) $.
    [1 punt]
    \item Determineu el valor de $ x $ que fa que el volum sigui màxim i calculeu aquest volum màxim.
    [1 punt]
\end{enumerate}

\textbf{6. Problem 6}
Siguin les paràboles $ f(x) = x^{2} + k^{2} $ i $ g(x) = -x^{2} + 9 k^{2} $.
\begin{enumerate}
    \item Calculeu les abscisses, en funció de $ k $, dels punts d'intersecció entre les dues paràboles.
    [1 punt]
    \item Calculeu el valor del paràmetre $ k $ perquè l'àrea compresa entre les paràboles sigui de 576 unitats quadrades.
    [1 punt]
\end{enumerate}

\subsubsection*{Sèrie 5}

\textbf{1. Problem 1}
Considereu el sistema d'equacions lineals $ \left(\begin{array}{ccc}1 & 1 & 2 \\ -4 & -1 & -5 \\ 3 & 1 & 4\end{array}\right)\left(\begin{array}{l}x \\ y \\ z\end{array}\right)=\left(\begin{array}{c}b_{1} \\ b_{2} \\ b_{3}\end{array}\right) $.
\begin{enumerate}
    \item Expliqueu raonadament si les afirmacions següents són vertaderes o falses:
    \begin{enumerate}
        \item Si $ \left(\begin{array}{c}b_{1} \\ b_{2} \\ b_{3}\end{array}\right) = \left(\begin{array}{c}0 \\ 0 \\ 0\end{array}\right) $, el sistema és compatible determinat i la solució és $ \left(\begin{array}{l}x \\ y \\ z\end{array}\right) = \left(\begin{array}{c}0 \\ 0 \\ 0\end{array}\right) $.
        [1 punt]
        \item Si $ \left(\begin{array}{c}b_{1} \\ b_{2} \\ b_{3}\end{array}\right) = \left(\begin{array}{l}1 \\ 1 \\ 1\end{array}\right) $, el sistema és compatible indeterminat.
        [1 punt]
    \end{enumerate}
\end{enumerate}

\textbf{2. Problem 2}
Siguin a $ \mathbb{R}^{3} $ el pla $ \pi $ d'equació $ x - y + 2 z = 2 $ i els punts $ A = (3, -1, 2) $ i $ B = (1, 1, -2) $.
\begin{enumerate}
    \item Comproveu que els punts $ A $ i $ B $ són simètrics respecte del pla $ \pi $.
    [1 punt]
    \item Si $ r $ és la recta dels punts $ P $ que té per equació $ P = B + \lambda v $, en què $ \lambda $ és un paràmetre real i $ v = (1, 1, 0) $, verifiqueu que els punts mitjans dels segments $ \overline{A P} $ pertanyen al pla $ \pi $.
    [1 punt]
\end{enumerate}

\textbf{3. Problem 3}
\begin{enumerate}
    \item Calculeu els màxims relatius, els mínims relatius i els punts d'inflexió de la funció $ f(x) = 2 x^{3} - 9 x^{2} + 12 x - 4 $.
    [1 punt]
    \item Expliqueu raonadament que si $ f(x) $ és una funció amb la derivada primera contínua en l'interval $ [a, b] $ i satisfà que $ f'(a) > 0 $ i $ f'(b) < 0 $, aleshores hi ha, com a mínim, un punt de l'interval $ (a, b) $ en què la recta tangent a la gràfica de $ f(x) $ en aquest punt és horitzontal.
    [1 punt]
\end{enumerate}

\textbf{4. Problem 4}
Sigui $ \boldsymbol{A} $ una matriu quadrada d'ordre $ n $ que satisfà la igualtat $ \boldsymbol{A} \cdot (\boldsymbol{A} - \boldsymbol{I}) = \boldsymbol{I} $, en què $ \boldsymbol{I} $ és la matriu identitat.
\begin{enumerate}
    \item Justifiqueu que la matriu $ \boldsymbol{A} $ és invertible i que $ \boldsymbol{A}^{-1} = \boldsymbol{A} - \boldsymbol{I} $.
    [1 punt]
    \item Calculeu el valor de $ a $ que fa que la matriu $ \boldsymbol{A} = \left(\begin{array}{ll}1 & 1 \\ 1 & a\end{array}\right) $ compleixi la igualtat $ \boldsymbol{A} \cdot (\boldsymbol{A} - \boldsymbol{I}) = \boldsymbol{I} $. Calculeu $ \boldsymbol{A}^{-1} $ i comproveu que es correspon amb la matriu calculada a partir del resultat de l'apartat anterior.
    [1 punt]
\end{enumerate}

\textbf{5. Problem 5}
Siguin les rectes $ r: (x, y, z) = (2, 3, -3) + \lambda (1, -1, 0) $ i $ s: \frac{x-3}{2} = y - 5 = z + 2 $.
\begin{enumerate}
    \item Estudieu si les rectes $ r $ i $ s $ són paral·leles o perpendiculars.
    [1 punt]
    \item Determineu la posició relativa de les rectes $ r $ i $ s $ i calculeu l'equació paramètrica de la recta $ t $ que talla perpendicularment la recta $ r $ i la recta $ s $.
    [1 punt]
\end{enumerate}

\textbf{6. Problem 6}
Sabem que una funció $ f(x) $ té per derivada $ f'(x) = (x + 1) \mathrm{e}^{x} $ i que $ f(0) = 2 $.
\begin{enumerate}
    \item Trobeu l'equació de la recta tangent a $ y = f(x) $ en el punt de la corba d'abscissa $ x = 0 $.
    [1 punt]
    \item Calculeu l'expressió de $ f(x) $.
    [1 punt]
\end{enumerate}

\end{document}