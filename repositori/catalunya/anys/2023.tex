\documentclass{article}
\usepackage{amsmath}

\begin{document}

\subsection*{Year: 2023}

\subsubsection*{Sèrie 1}

\textbf{1. Problem 1}
Calculeu els coeficients \( a, b, c \) i \( d \) de la funció \( f(x) = a x^3 + b x^2 + c x + d \) si sabem que l'equació de la recta tangent a la gràfica de la funció \( f \) en el punt d'inflexió \( (1,0) \) és \( y = -3 x + 3 \) i que la funció té un extrem relatiu en el punt de la gràfica d'abscissa \( x = 0 \).
[2,5 punts]

\textbf{2. Problem 2}
Considereu les dues matrius següents:
\[
\boldsymbol{A} = \left(\begin{array}{ccc}
2 & -3 & -5 \\
-1 & 4 & 5 \\
1 & -3 & -4
\end{array}\right) \quad \boldsymbol{B} = \left(\begin{array}{ccc}
2 & 2 & 0 \\
-1 & -1 & 0 \\
1 & 2 & 1
\end{array}\right)
\]
\begin{enumerate}
    \item Calculeu les matrius \( \boldsymbol{A} \cdot \boldsymbol{B} \) i \( \boldsymbol{B} \cdot \boldsymbol{A} \).
    [1,5 punts]
    \item Siguin \( \boldsymbol{C} \) i \( \boldsymbol{D} \) dues matrius quadrades del mateix ordre que satisfan \( \boldsymbol{C} \cdot \boldsymbol{D} = \boldsymbol{C} \) i \( \boldsymbol{D} \cdot \boldsymbol{C} = \boldsymbol{D} \). Comproveu que les dues matrius, \( \boldsymbol{C} \) i \( \boldsymbol{D} \), són idempotents.
    [1 punt]
    \item Nota: Una matriu quadrada s'anomena idempotent si coincideix amb el seu quadrat.
\end{enumerate}

\textbf{3. Problem 3}
Sigui \( f'(x) = \left\{\begin{array}{ll}x-1, & \text { si } x \leq 2 \\ \frac{1}{x-1}, & \text { si } x>2\end{array}\right. \) la función derivada d'una función derivable \( f(x) \) que passa pel punt \( A = (0,3) \).
\begin{enumerate}
    \item Calculeu la función \( f(x) \).
    [1,5 punt]
    \item Calculeu l'equació de la recta tangent a la función \( f'(x) \) en el punt d'abscissa \( x = 3 \). [1 punt]
\end{enumerate}

\textbf{4. Problem 4}
Sigui el sistema d'equacions lineals segünt, que depèn del paràmetre real \( \lambda \):
\[
\left\{\begin{array}{l}
x + 2 \lambda y + (2 + \lambda) z = 0 \\
(2 + \lambda) x + y + 2 \lambda z = 3 \\
2 \lambda x + (2 + \lambda) y + z = -3
\end{array}\right.
\]
\begin{enumerate}
    \item Discutiu el sistema per als diferents valors del paràmetre \( \lambda \).
    [1,25 punts]
    \item Per al cas \( \lambda = -1 \), resoleu el sistema, interpreteu-lo geomètricament i identifiqueu-ne la solució.
    [1,25 punts]
\end{enumerate}

\textbf{5. Problem 5}
La Núria té un jardí rectangular i vol fer-hi un tancat (rectangular o quadrat) de \( 8 \, \text{m}^2 \) per al seu gos. Ha pensat de posar el tancat tocant al mur del jardí, tal com es mostra a la figura de la dreta, per estalviar-se així un dels quatre costats.

El preu de la tanca que vol fer servir és de \( 2,5 \, € / \text{m} \).
\begin{enumerate}
    \item Quines dimensions ha de tenir el tancat perquè el cost sigui mínim? Quin és aquest cost mínim?
    [1,75 punts]
    \item Si manteniu la forma rectangular o quadrada del tancat i feu que un dels vèrtexs del jardí coincideixi amb un vèrtex del tancat, quants euros us podeu estalviar? Raoneu com posaríeu el tancat i justifiqueu amb càlculs matemàtics les dimensions de la vostra proposta.
    [0,75 punts]
\end{enumerate}

\textbf{6. Problem 6}
Siguin els plans \( \pi_1 \) i \( \pi_2 \), determinats respectivament per les equacions \( \pi_1: x + y = 3 \) i \( \pi_2: x - z = -2 \).
\begin{enumerate}
    \item Trobeu l'equació general \( (A x + B y + C z + D = 0) \) del pla \( \pi_3 \), que és perpendicular a \( \pi_1 \) i \( \pi_2 \), i que passa pel punt \( P = (4, 1, 2) \).
    [0,75 punts]
    \item Sigui \( r \) la recta d'intersecció de \( \pi_1 \) i \( \pi_2 \). Calculeu l'equació vectorial de la recta \( r \).
    [0,75 punts]
    \item Calculeu el punt \( Q \) de la recta \( r \) que és més a prop del punt \( P \).
    [1 punt]
\end{enumerate}

\subsubsection*{Sèrie 5}

\textbf{1. Problem 1}
Considereu les funcions \( f(x) = -x^2 + x + 6 \) i \( g(x) = -9 x + 3 x^2 \).
\begin{enumerate}
    \item Calculeu l'àrea de la región delimitada per les dues funcions.
    [1,25 punts]
    \item Trobeu l'equació de la recta tangent a la funció \( f(x) \) en el punt \( (-2, 0) \). Representeu aquesta recta tangent i les funcions \( f(x) \) i \( g(x) \) en uns mateixos eixos de coordenades. [1,25 punts]
\end{enumerate}

\textbf{2. Problem 2}
Considereu el sistema d'equacions lineals
\[
\left.\begin{array}{c}
x - y + k z = -1 \\
x + k y + z = 3 \\
2 x + (k - 1) y + 2 z = 2
\end{array}\right\}
\]
en què \( k \) és un paràmetre real.
\begin{enumerate}
    \item Discutiu el sistema en funció del valor de \( k \).
    [1,5 punts]
    \item Resoleu el sistema per a \( k = 0 \) i per a \( k = 1 \).
    [1 punt]
\end{enumerate}

\textbf{3. Problem 3}
Considereu les rectes a l'espai \( r: x = -y = z + m \) i \( s: \left\{\begin{array}{c}x + y = 1 \\ x - z = 0\end{array}\right\} \), en què \( m \) és un paràmetre real.
\begin{enumerate}
    \item Estudieu la posición relativa per als diferents valors del paràmetre \( m \).
    [1,25 punts]
    \item Calculeu \( m \) perquè la distància entre les rectes \( r \) i \( s \) sigui de \( \sqrt{2} \) unitats. [1,25 punts]
\end{enumerate}

\textbf{4. Problem 4}
En una carretera principal hi trobem el poble \( A \). A 12 km del poble \( A \), hi ha un encreuament \( O \) amb una carretera secundària que talla perpendicularment la carretera principal. A 9 km de l'encreuament, a la carretera secundària, hi trobem el poble \( B \). Es vol construir una torre de comunicacions \( T \) en un punt de la carretera principal situat entre el poble \( A \) i l'encreuament \( O \). Aquesta torre ha d'estar connectada amb cadascun dels dos pobles en línia recta per cable. Sabem que instal·lar el cable entre la torre \( T \) i el poble \( B \) té un preu de \( 250 \, € / \text{km} \) i, en canvi, instal·lar el cable entre la torre \( T \) i el poble \( A \) té un preu de \( 125 \, € / \text{km} \). Determineu a quina distància de l'encreuament \( O \) a la carretera principal cal situar la torre \( T \) perquè el preu del cablejat sigui mínim i quin serà el valor d'aquest preu mínim.
[2,5 punts]

\textbf{5. Problem 5}
Considereu la família \( S \) de matrius de la forma \( \left(\begin{array}{ll}a & b \\ b & a\end{array}\right) \), en què \( a, b \in \mathbb{R} \).
\begin{enumerate}
    \item Calculeu \( \left(\begin{array}{ll}2 & 3 \\ 3 & 2\end{array}\right) \cdot \left(\begin{array}{ll}2 & 1 \\ 1 & 2\end{array}\right)^{-1} \).
    [1,25 punti]
    \item Trobeu totes les matrius de la família \( S \), és a dir, de la forma \( \boldsymbol{A} = \left(\begin{array}{ll}a & b \\ b & a\end{array}\right) \), que verifiquin la igualtat \( \boldsymbol{A}^2 = \boldsymbol{I} \), en què \( \boldsymbol{I} \) és la matriu identitat d'ordre 2.
    [1,25 punts]
\end{enumerate}

\textbf{6. Problem 6}
Sigui la función \( f(x) = \frac{a x^2 + x + b}{x^2 + 1} \).
\begin{enumerate}
    \item Calculeu els valors dels paràmetres \( a \) i \( b \) si sabem que la gràfica de la función \( f \) té un extrem relatiu en \( x = -1 \) i passa pel punt \( P = \left(-2, \frac{13}{5}\right) \).
    [1,25 punts]
    \item Per al cas \( a = b \), calculeu i classifiqueu els extrems relatius de la función. [1,25 punts]
\end{enumerate}

\end{document}
